\documentclass{beamer}
\usetheme{Boadilla}
\usepackage{tikz}
% \usepackage{enumitem}

\title{MTH 1020 Week 10 tutorial}
\author{Alex Elzenaar}

\newcommand{\R}{\mathbb{R}}

\begin{document}
\begin{frame}{\inserttitle}
\begin{enumerate}
  \item Work through a problem from last week
  \item Teaching evaluations
  \item
\end{enumerate}
\end{frame}


\begin{frame}{Week 9, Q6(b)}
  \begin{block}{Question}
    Find all local maxima and minima of $ f(t) = \sqrt{2-t} + t $.
  \end{block}

  \begin{itemize}
    \item \textbf{What is the domain of $ f $?}
    $f$ is defined everywhere that $ \sqrt{2-t} $ is defined, which is the interval $ (-\infty,2] $. So $ t = 2 $ is a critical point.
    \item \textbf{Where is the derivative zero?}
    Using the chain rule we find
    \begin{displaymath}
      f'(t) = -\frac{1}{2\sqrt{2-t}} + 1.
    \end{displaymath}
    So $ f'(t) = 0 \iff 2\sqrt{2-t} = 1 \iff 2-t = 1/4 \iff t = 7/4 $. So $ t = 7/4 $ is a critical point.
    \item \textbf{Where is the derivative undefined?}
    $f'(t) $ is defined everywhere $ \sqrt{2-t} \neq 0 $, which is everywhere except $ t = 2 $ (on the edge of the domain anyway).
  \end{itemize}
\end{frame}


\begin{frame}{Week 9, Q6(b)}
  \begin{block}{Question}
    Find all local maxima and minima of $ f(t) = \sqrt{2-t} + t $.
  \end{block}
  \begin{itemize}
    \item We computed
          \begin{displaymath}
            f'(t) = -\frac{1}{2\sqrt{2-t}} + 1
          \end{displaymath}
          and found two critical points, $ t = 2 $ and $ t = 7/4 $.
    \item The second derivative is
    \begin{displaymath}
      f''(t) = -\frac{1}{4 (2-t)^{-3/2}}.
    \end{displaymath}
    \item At $ t = 7/4 $, $ f''(t) = -2 $. So the function is concave down and $ t=2 $ is a local max.
    \item $ f''(2)$ is undefined so we need a different argument. Note that $ f''(t) $ is defined in the
            entire open interval $ (-\infty,2) $ and $ f'(t) $ is negative when $ t > 7/4 $; in particular
            as $ t \to 2 $, the function $ f $ is decreasing, so $ t=2 $ is a local minimum.
  \end{itemize}
\end{frame}


\begin{frame}{Key concepts so far}
\begin{enumerate}
  \item Limit calculations. Definitions of continuity, differentiability.
    \begin{itemize}
      \item \textbf{Being able to compute simple limits is a `pass' skill.}
      \item Know different interpretations of derivative: rate of change, tangent line slope.
      \item Have a zoo of weird examples e.g. $ x\sin(1/x) $, $ \lvert x \rvert $,...
    \end{itemize}
  \item Usual differentiation laws (chain and product laws, implicit differentiation) and derivatives of basic functions ($ x^n $, $ \sin x $, $ \cos x  $, $ \exp x $, $ \log x $, $ \lvert x \rvert $).
    \begin{itemize}
      \item Do enough examples that you don't need to think to compute $ f' $ from a formula for $ f $.
      \item \textbf{Being able to compute derivatives mechanically is a `pass' skill.}
    \end{itemize}
  \item Intermediate value theorem: if $ f : [a,b] \to \R $ is a continuous function, $ f(a) < 0 $, and $ f(b) > 0 $, then there exists $ c \in [a,b] $ such that $ f(c) = 0 $
  \item Extrema: definition of critical point, inflection point, concavity. Optimisation problems.
\end{enumerate}
\end{frame}


\begin{frame}{Teaching evaluations}
\begin{enumerate}
  \item I will hand out the evaluation sheets and then leave for 5 minutes to let you fill it in (feel free to talk amongst yourselves).
  \item \textbf{Please keep the sheets anonymous. Do not write anything personally identifiable. Definitely do NOT put your name on the sheet.}
  \item When I will come back I will get a volunteer to go around with an envelope for you to put your sheets into. Please fold them in half
  so that the writing is not visible.
  \item \textbf{The sheets will go to Simon to read, I don't get them until the end of the semester when all the grades are finalised.} Please bear this in mind,
  if there is something specific you want to see changed during the semester it is better to talk to me or Simon directly if you are comfortable doing that, otherwise
  the feedback will only help us next time we teach.
\end{enumerate}
\end{frame}



\begin{frame}{\inserttitle}
\begin{enumerate}
  \item Get into groups of 3-4 people who all prepared a different question in advance.
  \item Write your \textbf{preferred name} and \textbf{ID number} on the whiteboards so I can take attendance
  \item Present your prepared question to each other as I come around, you should only take about 5min each for this.
  \item Then get started on the other questions \textbf{in your groups}.
  \item \textbf{At the end:} please erase the boards and return any markers etc that you used (you do not need to return the handouts)
\end{enumerate}
\end{frame}



\end{document}
