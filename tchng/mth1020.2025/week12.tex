\documentclass{beamer}
\usetheme{Boadilla}
\setbeamertemplate{footline}{}
\usepackage{tikz}
\usepackage{physics}
% \usepackage{enumitem}

\title{MTH 1020 Week 12 tutorial}
\author{Alex Elzenaar}

\newcommand{\R}{\mathbb{R}}

\begin{document}
\begin{frame}{\inserttitle}
\begin{enumerate}
  \item Three worked integrals
  \item Integration
\end{enumerate}
\end{frame}

\begin{frame}
  \begin{block}{Theorem (Archimedes)}
    The volume of the cone with height $ h $ and radius $ r $ is one-third the volume of the cylinder it lies in.
  \end{block}
  \begin{proof}
    Slice the cone into discs.

    By similar triangles, the disc at distance $ t $ from the tip of the cone has radius $ t/h \cdot r $. Hence it has area $ \pi (\frac{tr}{h})^2 $.

    Therefore the volume of the cone is
    \begin{displaymath}
      \int_{0}^h \pi \left(\frac{tr}{h}\right)^2 \,dt = \frac{\pi r^2}{h^2} \int_0^h t^2\,dt = \frac{\pi r^2}{h^2} \cdot \frac{1}{3} h^3 = \frac{\pi}{3} r^2 h.
    \end{displaymath}
  \end{proof}
\end{frame}

\begin{frame}
  \begin{block}{Integral 1}
    \begin{displaymath}
      \int \frac{1}{1+x^2} \, dx
    \end{displaymath}
  \end{block}
  \pause
  \textbf{Method 1.} Look at it and immediately see, `it is the derivative of $\arctan$'.

  \pause\vspace{1ex}

  \textbf{Method 2.} $ 1+x^2 $ looks like it might simplify using a trig identity: remember that $ \tan^2 + 1 = \sec^2 $.

  \pause
  Let $ x = \tan \theta $, so $ dx = \sec^2 \theta \, d\theta$. \pause Then
  \begin{align*}
    \int \frac{1}{1+x^2} \, dx &= \int \frac{1}{1+\tan^2 \theta} \sec^2 \theta  \, d\theta \\
                               &= \int \frac{\sec^2 \theta}{\sec^2 \theta}  \, d\theta \\
                               &= \int\, d\theta = \theta = \arctan x.
  \end{align*}
\end{frame}

\begin{frame}
  \begin{block}{Integral 2}
    \begin{displaymath}
      \int \arcsin x \, dx
    \end{displaymath}
  \end{block}
  \pause
  \textbf{Method 1.} Try substituting the only thing possible: $ \theta = \arcsin x $. This means $ x = \sin \theta $
  and so $ dx = \cos \theta \, d\theta $.  \pause The integral becomes
  \begin{displaymath}
    \int \arcsin x \, dx = \int \theta \cos \theta \, d\theta
  \end{displaymath}
  \pause
  This should look like an integration by parts thing. Take
  \begin{displaymath}
    \begin{array}{ll}
      u = \theta & \uncover<5->{v = \sin\theta}\\
      \uncover<5->{du = \, d\theta} & dv = \, \cos\theta \, d\theta
    \end{array}
  \end{displaymath}\pause\pause
  and we end up with
  \begin{align*}
    \int \theta \cos \theta \, d\theta &= \theta \sin \theta - \int \sin \theta \, d\theta\\
                                       &= \theta \sin \theta + \cos \theta\\
                                       &= x \arcsin x + \cos \arcsin x\\
                                       &= x \arcsin x + \sqrt{1-x^2}.
  \end{align*}
\end{frame}

\begin{frame}
  \begin{block}{Integral 2}
    \begin{displaymath}
      \int \arcsin x \, dx
    \end{displaymath}
  \end{block}
  \pause
  \textbf{Method 2.} Try integration by parts straight away: let
  \begin{displaymath}
    \begin{array}{ll}
      u = \arcsin x & \uncover<3->{v = x}\\
      \uncover<3->{du = (1-x^2)^{-1/2} \, dx} & dv = \, dx
    \end{array}
  \end{displaymath}
  \pause\pause
  so our integral becomes
  \begin{equation}\tag{*}
    \int \arcsin x \, dx = x\arcsin x - \int \frac{x}{\sqrt{1-x^2}} \,dx.
  \end{equation}
  \pause
  Let $ w = 1-x^2 $, so $ dw = -2x \, dx $ and
  \begin{align*}
    \int \frac{x}{\sqrt{1-x^2}} \,dx &= \int \frac{\frac{1}{-2x} x }{\sqrt{w}} \, dw\\
                                     &= -\frac{1}{2}\int \frac{1}{\sqrt{w}} \, dw\\
                                     &= -\sqrt{w} = -\sqrt{1-x^2}.
  \end{align*}
\end{frame}

\begin{frame}
  \begin{block}{Integral 2}
    \begin{displaymath}
      \int \arcsin x \, dx
    \end{displaymath}
  \end{block}
  \textit{\color{gray}....integrated by parts...}
  \begin{equation}\tag{*}
    \int \arcsin x \, dx = x\arcsin x - \int \frac{x}{\sqrt{1-x^2}} \,dx.
  \end{equation}
  Let $ w = 1-x^2 $, so $ dw = -2x \, dx $ and
  \begin{align*}
    \int \frac{x}{\sqrt{1-x^2}} \,dx &= \int \frac{\frac{1}{-2x} x }{\sqrt{w}} \, dw\\
                                     &= -\frac{1}{2}\int \frac{1}{\sqrt{w}} \, dw\\
                                     &= -\sqrt{w} = -\sqrt{1-x^2}.
  \end{align*}
  \pause
  Substituting this back into (*) gives
  \begin{displaymath}
    \int \arcsin x \, dx = x\arcsin x + \sqrt{1-x^2}.
  \end{displaymath}
\end{frame}

\begin{frame}{Key concepts (integration)}
\begin{enumerate}
  \item Definite integral as a Riemann sum. Indefinite integral as anti-differentiation.
  \item Fundamental theorem of calculus.
  \item Area calculation.
  \item Techniques of integration:
    \begin{enumerate}
      \item Substitution (chain rule)
      \item By parts (product rule)
      \item By partial fractions
      \item By trig identities
    \end{enumerate}
\end{enumerate}
  \begin{block}{A special challenge integral}
    \begin{displaymath}
      \int \sqrt{\tan x} \, dx
    \end{displaymath}
  \end{block}
\end{frame}




\begin{frame}{\inserttitle}
\begin{enumerate}
  \item Get into groups of 3-4 people who all prepared a different question in advance.
  \item Write your \textbf{preferred name} and \textbf{ID number} on the whiteboards so I can take attendance
  \item Present your prepared question to each other as I come around, you should only take about 5min each for this.
  \item Then get started on the other questions \textbf{in your groups}.
  \item \textbf{At the end:} please erase the boards and return any markers etc that you used (you do not need to return the handouts)
\end{enumerate}
\end{frame}



\end{document}
